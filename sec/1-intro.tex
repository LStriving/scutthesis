\chapter{绪论}
% 简要说明研究工作的目的、范围、相关领域的前人工作和知识空白、理论基础、研究设想、研究方法和实验设计、预期结果和意义等。
%应言简意赅,不要与摘要雷同,不要写成摘要的注释。一般教科书中有的知识,在绪论中不必赘述。

本章首先介绍论文的研究背景与意义,然后对本文的研究内容与主要贡献进行阐
述,并对全文的组织结构进行介绍。

\section{研究背景和意义}

医疗健康领域的视频是海量视频数据最具价值的组成部分,主要包括面向辅
助诊断和评估的特殊造影视频(如心脏超声造影、吞咽造影),面向
康复医疗的行为视频,以及作为示教材料的手术视频等。在辅助诊断中,
相比静态图像,视频等动态影像能够客观具体地还原器官运动过程,从而为辅助
诊断提供关键动态信息,可大大提高诊断精度。例如,吞咽造影检查被认
为是吞咽障碍诊断的“金标准”。但是,由于缺乏有效针对性分析技术,
当前视频辅助诊断还主要依靠医生人工逐帧观察和定性分析。然而,受限于
医生自身经验和工作状态,既耗时费力,又难以保证客观性和准确性。实现
医疗视频智能分析和理解,对提升医疗辅助诊断水平、减轻医生工作量,提升全
社会医疗效率,意义重大。


\section{研究内容与贡献}

本文研究了类别不平衡的在线分类问题和在线主动学习问题。主要贡献可概括如下:
1) 针对类别不平衡数据流的分类算法研究:

\begin{figure*}[ht]
    \centering
    \includegraphics[width=0.8\linewidth]{fig/研究思路.png}
    \caption{本文的研究思路与研究内容}
    \label{fig: research_flow}
\end{figure*}

•本文提出了一种自适应代价敏感在线分类算法。通过对不同类别设置差异化的误分
类代价,该算法能够有效区别少数类样本和多数类样本,从而较好地解决类别不平衡。
同时,通过探索样本二阶信息,该算法能够提升模型收敛速度,更快速地适应时序数据
的分布变化,从而更好地处理类别不平衡数据流的分类问题。

•本文理论分析了所提出算法的性能边界和收敛速率。理论结果表明:相较于一阶在
线分类算法,所提出方法有更快的收敛速率。考虑到所使用的目标函数并非是强凸函数,
这一收敛速率已经达到了理论最优。

•本文在多个数据集上验证了所提出方法的有效性和优越性。实验结果表明:引入样
本二阶信息和代价敏感目标函数,能够有效提升在线分类算法在实际任务中的性能。除
此之外,本文还用大量实验检验算法的参数敏感度。

2) 针对类别不平衡数据流的主动学习算法研究:

•提出了一种在线自适应非对称在线主动学习算法。通过在模型更新和样本质询时对
不同类别设置差异化的权重,该算法能够有效区别少数类样本和多数类样本的重要性,
从而更多地质询少数类样本的标签,并训练一个对类别不平衡敏感的分类算法。同时,
通过探索样本二阶信息,该算法能够提升收敛速度,并解决模型训练早期的不稳定性,
从而快速地适应时序数据的分布变化,并做出更准的质询决定。因此,所提出算法能更
快更好地处理类别不平衡数据流的在线主动学习问题。

•在质询预算盈余和预算耗尽的两种情况下,本文分别分析了所提出算法的期望预测
错误数和性能边界。理论结果展现了所提出算法的有效性,并为其他在线主动学习算法
的理论研究提供了新的分析思路。

•本文实验验证了所提出方法的有效性和优越性。实验结果表明:同时在模型更新和
标签质询时考虑类别不平衡问题,并引入样本二阶信息,能够有效提升在线主动学习算
法在实际任务中的性能,具有重要应用意义。另外,本文还通过大量实验检验算法的稳
定度和参数敏感度。


\section{本文的组织结构}

本文的组织结构和章节关系安排如下:

第一章是绪论部分,介绍了本文的研究背景和意义,并阐述了研究内容和主要贡献。

第二章是相关工作综述,介绍了目前国内外在医疗场景下的视频动作理解研究现状,并分析了现有研究的不足之处。

第三章

第四章